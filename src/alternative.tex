In \cite{honeybee}, Dhinesh Babu L.D. and P. Venkata Krishna propose a honey bee
behaviour inspired load balancing algorithm.
This algorithm takes priority of tasks into account and works well for
heterogeneous cloud environments.
It also reduces the waiting time of tasks in the queues on the worker instances.

The Honey bee inspired load balancing algorithm is, as the name tells, based on
the foraging behaviour of honey bees and the \ac{ABC} as proposed by
N. Karaboga \cite{honeybee}.

Dhinesh Babu L.D. and P. Venkata Krishna illustrate that their algorithm
improves average execution time and reduce waiting time of queued tasks.

A. Nakai, E. Madeira and L. E. Buzato outline a server-based load balancing
policy for worldwide distributed web servers in \cite{nakai}.

Compared to known solutions (\ac{RR}, \ac{RR-AA}, \ac{SL} and \ac{LU} as
implemented in the ``HAProxy'' open source load balancing solution), their
algorithm presented better performance due to the fact that their algorithm
prevented overloading rather than rebalancing work if overloading occurs.

\cite{cloudLBTech} lists more load balancing techniques, featuring but not
limited to
\begin{itemize}
    \item \emph{Decentralized content aware load balancing}
    \item \emph{Load Balancing strategy for virtual storage}
    \item \emph{Load Balancing mechanism based on ant colony and complex network
        theory}
    \item \emph{Two-Phased Scheduling}
    \item \dots
\end{itemize}

