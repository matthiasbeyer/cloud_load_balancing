In the following chapter, we focus on how load balancing algorithms must work.
In the first section we describe what input data must be provided for the
algorithm to work.
In the next section we describe the properties of such an algorithm.
Finally, we compare some simple load balancing algorithms to display their
behaviour.

% algorithms chapter
\subsection{Input Data}
\label{sec:algo:input}

As \ac{LBA} should be implemented as \emph{pure} algorithms,
the input data of a \ac{LBA} is the
critical part of such an algorithm: If too few data points are
provided, the \ac{LBA} might return inaccurate values and the load balancing
does not meet the properties it wants to hold.
If too much data is provided, this might result in unnecessary overhead in the
calculation of the result.

The simplest \ac{LBA} has exactly one information it must know to work: How many
nodes it can distribute the data on.
Every time a new tasks is put into the \ac{LBA} to be distributed, the \ac{LBA}
can now schedule the task to another node, resulting in a ``Round-Robin''
(\ref{sec:algo:comp:roundrobin}) algorithm.
Such an algorithm might be feasible in an environment where the tasks as well as
the infrastructure to process the tasks are homogenous.
That is, when all tasks have the very same system requirements as well as all
computing nodes have the same specifications in available Memory and CPU.
Such a scenario is hardly common.

As tasks are not homogenous, more information is required on both system
requirements of each task as well as system specifications of the computing
nodes.
A \ac{LBA} should be fed with the global system state, so it can calculate the
best fitting node for the task to schedule.
Hence, a book-keeping mechanism should be implemented along the algorithm
itself.
The more information available about a task and the global state of the
machines, the better the algorithm can decide, although estimating the costs of
a task might be a challenge.

% Nodes + their system data (CPU, RAM, NETWORK connection)
% Kind of request (CPU intensive, RAM intensiev, network intensive)
% Kind of infrastructure: homogenouse vs. heterogenouse

\subsection{Properties of \ac{LBA}}
\label{sec:algo:prop}

The properties of a \ac{LBA} are, obviousely, that it must spread the task load
as evenly as possible over a set of computing nodes without taking a long time
to compute the destination of a task.

A \ac{LBA} should compute reproducible results.
That means, putting a set of computing nodes and a set of tasks into a load
balancing algorithm should result in the same distribution of the tasks if done
multiple times.

% * Spreading load evenly
% * Reproducible results
% * Statelessness
% * Speed
% * Reliability
%
% All related:
% * statistics
% * dynamic/automatic spawning of instances
% * hot-plugging instances
% * On-line upgrading

\subsection{Comparation of algorithms}

For the following comparation of different \ac{LBA} the author implemented all
of the mentioned algorithms in the Rust programming language.
The input data for all algorithms was generated randomly.

\label{sec:algo:comp}
