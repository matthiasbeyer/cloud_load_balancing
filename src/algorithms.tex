The complexity of a \ac{LBA} increases with the complexity of the system it has
to work with.
Consider a load balancing algorithm that computes the target worker for a task,
where the size of the tasks are equal, the workers have the same capabilities
and the number of overall tasks is known.
Such an algorithm is not of complexity.
In fact, it can be implemented by a simple mathematical equation:
\begin{equation}
    i = j \% |W|
\end{equation}
Where $i$ is the number of the worker instance in set $W$ that will be assigned
with the task number $j$.
Though, such an environment is unlikely.

In todays systems, the size of a task cannot be predicted.
The capabilities of a worker instance may be known, but perfect book-keeping is
hard, even more in a distributed system.

