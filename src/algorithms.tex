The complexity of a \ac{LBA} increases with the complexity of the system it has
to work with.
Consider a load balancing algorithm that computes the target worker for a task,
where the size of the tasks are equal, the workers have the same capabilities
and the number of overall tasks is known.
Such an algorithm is not of complexity.
In fact, it can be implemented by a simple mathematical equation:
\begin{equation}
    i = j \% |W|
\end{equation}
Where $i$ is the number of the worker instance in set $W$ that will be assigned
with the task number $j$.
Though, such an environment is unlikely.

In todays systems, the size of a task cannot be predicted.
The capabilities of a worker instance may be known, but perfect book-keeping is
hard, even more in a distributed system.

A cheap load balancing strategy is randomizing the target.
\begin{equation}
    i = rand() \% |W|
\end{equation}
Which has several advantages over other approaches:
\begin{itemize}
    \item Simplicity
    \item Few edge cases
    \item Easy failover
    \item Works distributed
\end{itemize} % TODO: cite http://www.codemesh.io/codemesh/tyler-mcmullen ?
Although, distributing load randomly does not guarantee the best distributing
over a given set of notes.

Another known load balancing algorithm is \ac{JSQ}.
With this algorithm, the load balancing mechanism implements some book-keeping
mechanism how many jobs are in the queues of the worker instances and selects
the instance with the shortest queue for a new task.
This algorithm can result in a ``herd effect'', where a subset of machines is
either underutilized or overloaded with tasks.
That is likely, if the load balancing instance(s) has old information on the
utilization of the nodes as described in \cite{inpSLoadInfo}.

A surprising optimization is the combination of the two algorithms described
above.
\begin{equation}
    \begin{aligned}
        x &= rand() \% |I| \\
        y &= rand() \% |I| \\
        i &= \begin{cases}
            x & \text{if } load(W_x) =< load(W_y)\\
            y & \text{if } load(W_x) > load(W_y)\\
        \end{cases}
    \end{aligned}
    \label{eq:randjsq}
\end{equation}
If we combine these algorithms, so we select two worker instances randomly and
now apply \ac{JSQ} on this new set of worker instances (see \ref{eq:randjsq}),
we get better distribution, as stated in \cite{powerOfTwoInRLB}.

