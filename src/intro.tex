``Load balancing'' itself is a board terminus used to describe the act of
distributing workload over a set of workers to increase the overall throughput.
In large scale computing environments such as ``the cloud'', load balancing can
be and is used to normalize workload of tasks and requests that are issued by
clients over the network
\cite{alakeel2010guide}.
Load balancing solutions can be categorized for example by implemented balancing
algorithm and network layer they operate on.

A \ac{LBA} must be able to distribute load as evenly as possible
over a set of nodes.
Additionally a \ac{LBA} itself must hold several properties,
such as
\begin{itemize}
    \item Reproducible results
    \item Statelessness
    \item Speed
    \item Reliability
\end{itemize}
Other desireable features are dynamic spawning of worker instances or
hot-plugging of instances, generation of statistics and possibly even on-line
upgrading.

The following chapters will explain why these properties are feasible.

